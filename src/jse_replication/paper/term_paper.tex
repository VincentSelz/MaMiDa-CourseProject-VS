% !TeX spellcheck = en_US
% !TeX document-id = {2870843d-1baa-4f6a-bd0a-a5c796104a32}
% !BIB TS-program = biber
% !TeX encoding = UTF-8

\documentclass[11pt,a4paper,leqno]{article}

\usepackage{a4wide}
\usepackage[T1]{fontenc}
\usepackage[utf8]{inputenc}
\usepackage{float, afterpage, rotating, graphicx}
\usepackage{epstopdf}
\usepackage{longtable, booktabs, tabularx}
\usepackage{fancyvrb, moreverb, relsize}
\usepackage{eurosym, calc}
% \usepackage{chngcntr}
\usepackage[flushleft]{threeparttable}
\usepackage{amsmath, amssymb, amsfonts, amsthm, bm}
\usepackage{caption}
\usepackage{mdwlist}
\usepackage{xfrac}
\usepackage{setspace}
\usepackage{xcolor}
\usepackage{subcaption}
\usepackage{minibox}
\usepackage{hyperref}
% \usepackage{pdf14} %Enable for Manuscriptcentral
% \usepackage{endfloat} %Enable to move tables / figures to the end. Useful for some submissions.

% BIB SETTINGS
\usepackage[
backend=biber,
giveninits=true,
maxnames=30,
maxcitenames=20,
uniquename=init,
url=false,
style=authoryear,
]{biblatex}
\addbibresource{refs.bib}
\setlength\bibitemsep{0.3cm} % space between entries in the reference list


\widowpenalty=10000
\clubpenalty=10000

\setlength{\parskip}{1ex}
\setlength{\parindent}{0ex}
\setstretch{1.5}

\title{Towards a replication of \textcite{MST2021}:  \\ Using the Survey of Consumer Expectations\thanks{Vincent Selz, Goethe University. Email: \texttt{vincent[dot]selz[at]stud[dot]uni-frankfurt[dot]de}. } }

\author{Vincent Selz \\ 7707241}

\date{
	\today
}

\begin{document}

\maketitle


\clearpage
\section{Introduction}

% Introduction 
% Data
% Data: Repository + Pytask
% Paper/Replication
% Contribution (Kommentar mit den Imputed vs. Standard)


In this term paper, I will replicate elements of \textcite{MST2021}. They leverage the Survey of Consumer's Expectation (henceforth SCE) to investigate job seekers perception of their employment prospects. In particular, they are interested in disentangling true duration dependence (Ljungbist, Sargent) from a selection effect where the observed negative duration dependence is driven by a selection effect. 

% Data Strategy
In order to promote transparency and replicability, the entire project has been stored on Github, which utilizes the workflow management system, \textit{pytask}, developed by \textcite{Raabe2020}.  Following the instructions provided on Github, it should be possible for anyone  to replicate the results. The replication process focuses on reproducing the reduced form results using the original data, and then comparing them to the outcomes obtained by using the SCE data downloaded from the Federal Reserve Bank of New York's website.

\autoref{sec:data} will introduce the SCE and will  follow the outline given to us for the presentations. Subsequently,
\autoref{sec:paper} will present my work regarding the replication of the reduced form results of the paper and \autoref{sec:conclusion} gives an outlook to future avenues of research and concludes.

\section{Data} \label{sec:data}
% Should be 2/3 pages and can be concluded with Table 1 from MST
% Quick Overview
The SCE is a monthly survey conducted by the Federal Reserve Bank of New York conducted since 2013 that aims to gather information on consumer expectations. \textcite{SCEOverview} provide a comprehensive overview over the SCE. The survey is based on a rotating panel of about 1,300 households, and respondents are asked a range of questions about their expectations for aggregate and individual economic conditions. The data collected from the survey is used to help inform monetary policy decisions and to better understand consumer behavior and economic trends.

% Details on the sample
The sample of the SCE is based on the Consumer Confidence Survey (CCS) which is a mail survey which selects a new random sample each month on the basis of the universe of U.S. Postal Service addresses and targets the household heads. The SCE consists of the subset of CCS respondents who indicated a willingness to participate in the SCE. One condition, however, is the availability of Internet and an email address. From the CCS respondents who signaled a willingness to participate, the sample is drawn via a stratified random sampling approach. This stratified approach spans the following dimensions: income, gender, race/ethnicity and Census Division.   Due to the internet based approach, people under the age of 30 are over sampled \parencite{SCEOverview}. Weights are provided to make the sample representative for the population of U.S. household heads. They are based on region, age, education and income and are targeting the Census population estimates from the American Community Survey. 

% SCE and MSC
The SCE and the Michigan Survey of Consumers both aim to collect data on consumer expectations, but the SCE focuses on gathering rich, high-frequency data. Compared to the Michigan Survey, the SCE has several advantages. Firstly, it covers a broader range of topics and gathers detailed information on areas such as expectations for income growth and home prices. Secondly, the SCE asks questions related to the probability of future events and can also measure uncertainty around the expectation. The broader range of topics covered and detailed information gathered by the SCE make it a more comprehensive tool for analyzing consumer expectations and their potential impact on the economy.

%  How to get the data % Data qualities/ Pecularities
The Federal Reserve Bank of New York's website publishes the results of the SCE on a monthly basis, which are available to the public without requiring a subscription.  All datasets, along with the survey questionnaire, are readily provided. The data is supplied in several excel-files which is convenient but has the disadvantage of consuming comparatively a lot of storage. The variable names are consistent across waves and available in a panel format. The data coding follows the specifications provided in the questionnaire and is easy to comprehend. In addition to the core module, the SCE offers several supplements such as the SCE Credit Access Survey, SCE Public Policy Survey and SCE Household Spending Survey. They are fielded quarterly as a rotating module of the SCE. 

% Density forecasts
Extending the methodology proposed by \textcite{Manski}, the SCE uses three types of questions to elicit probabilistic expectations. The first type of question asks about binary outcomes, such as whether the personal income will increase over the next twelve months. The second type of question asks for point estimates of continuous outcomes, such as the percentage increase or decrease in earnings over the next twelve months. The third type of question asks respondents to give density forecasts, where they have to assign percentage points over a range of bins for the continuous outcomes. Density forecasts provide information beyond just the first moments of the distribution, allowing for the measurement of uncertainty about economic conditions at the individual level. Leveraging this information can provide a new way to test macroeconomic models using micro data and to understand the impact of uncertainty on economic decisions. 

% This feels a little bit bumpy for now
\textcite{BFKL2018} utilize the SCE to empirically investigate the effect of uncertainty on economic choices. First, they show that there is substantial heterogeneity across households regarding expectations. Second, they document that  uncertainty in expectation can predict choices. In particular, they find that more uncertain expectations lead to precautionary behavior. Because in models where certainty equivalence is absent, decisions made today are influenced not only by the expectation of the future wage but also the uncertainty associated with it. Utilizing the subjective density forecasts, in contrast to the rational expectation, can yield a more detailed picture of economic decision-making. This can have implications for differential policies aimed at improving welfare. 

% Check das nochmal anständig und schreib es nochmal anständig.
In a similar vein, \textcite{Baleer2021} use the SCE to measure labor risk for various demographic groups. They find substantial optimistic bias which are heterogeneous throughout different groups. Based on this, they show that differences in biases can help understand savings behavior across individuals which is a quantitatively important channel in wealth inequality. 
% One more paper would be great
%

% Probably easiest to stick quite close to what they are saying about their strategy
\textcite{MST2021} use the elicited beliefs of job seeker's to investigate the negative duration dependence of unemployment. For this purpose, they use the following two questions administered to job seekers in the SCE:
\begin{center}
	\textit{What do you think is the percent chance that within the coming 12 months, you will find a job that you will accept, considering the pay and type of work?}
\end{center}
and
\begin{center}
	\textit{And looking at the more immediate future, what do you think is the percent chance that within the coming 3 months, you will find a job that you will accept, considering the pay and type of work?}
\end{center}

Along with other information, respondents in the study were asked to report their duration of unemployment. However, self-reports during unemployment spells can be inconsistent.
% Attrittion
 One advantage of the SCE is that it experiences low attrition. 58\% of all respondents complete all 12 monthly interviews \parencite{SCEOverview}. This allows the authors to construct unemployment durations themselves and thus avoid relying solely on possibly inconsistent self-reports.  Moreover, by tracking individuals over the course of a year, the SCE allows the authors to infer the realized exit rates from unemployment. 
 
     \begin{table}[!htbp] 
 	\centering 
 	\caption{Summary Statistics} 
 	\label{tab:summary_stats} 
 	\input{../../../bld/tables/tab1_summary_statistics_sce.tex}
 	\begin{minipage}[center]{0.7\textwidth}
 		\caption*{\footnotesize \textbf{Notes:} Survey weights are used for all calculations. (3) replicates the reported statistics from the paper exactly. (2) corrects for the error reported above and (1) is displaying the summary statistics for the SCE sample.}
 	\end{minipage}
 \end{table}
 
 %Summary statistics
 \autoref{tab:summary_stats}compares the summary statistics of the authors data to the data from the website. The sample used in the comparison is restricted to unemployed people aged between 20 and 65 who have answered the belief questions. However, it should be mentioned that there appears to be a minor error in their calculations. Specifically, when calculating the weighted proportions of the groups, they used the (id-date)-unit instead of the individual level. Hence, they count each individual as many time as the individual participated in the survey. Upon closer examination, it appears that the sample in the paper is younger and less educated than originally reported, with an even larger share of females. To investigate the extent of the differences between the data used by the authors and the data available on the SCE website, we compare (1) and (2). While the sample used by the authors is larger, the aggregate differences seem to be minor. In fact, the difference between the job finding rate for long-term unemployed and short-term unemployed seems to be even more pronounced in the data from the SCE website.  

 % Data differences
Based on the number of observations presented in \autoref{tab:summary_stats}, it seems that I was not able to obtain the exact same dataset as the authors. As shown in \autoref{fig:comp_over_time}, the authors had access to earlier data that is, as far as I know, not publicly available. This explains the discrepancies observed initially, although the number of responses in subsequent waves is quite similar. Unfortunately, we were unable to match respondents based on their IDs since the author's dataset uses a different ID structure that cannot be directly linked to the data available on the website.

\begin{figure}[!htbp] \centering
	\includegraphics[width=0.75\textwidth]{../../../bld/figures/data_comparison_over_time.png}
	\begin{minipage}[center]{0.75\textwidth}
		\caption*{\footnotesize \textbf{Notes:} The figure displays the number of respondents in each month for the authors data and the data from the website.}
	\end{minipage}
  \caption{Data Comparison Over Time}
\label{fig:comp_over_time}
\end{figure}

\section{Paper} \label{sec:paper}

% Theoretical Underpoinning
% Cite papers on both sides of the discussion 
Understanding the determinants of negative duration dependence of unemployment on the job finding rate remains a topic of long-standing discussion within labor economics.  One possible explanation for this phenomenon is that skills deteriorate over the course of an unemployment spell. Another possibility is that employers screen potential employees based on their duration of unemployment, resulting in lower job-finding rates for those who have been unemployed for longer periods. In this approach, the job-finding rate declines at the individual level over the unemployment spell. Alternatively, some scholars view duration dependence as a sorting mechanism, where more employable individuals find jobs more quickly, leaving the less employable individuals to select into long-term unemployment. However, the primary challenge in understanding duration dependence is that the job-finding rate is unobservable, and only the binary outcome whether someone finds a job or not can be observed. 

% Conceptual Framework
The authors approach this literature from a new angle. They leverage the elicited beliefs of job seeker's and combine them with actual job finding to identify the true heterogeneity of job finding and separate the dynamic selection effect from true (individual) duration dependence. Specifically, the covariance between the perceptions and the actual job finding can inform on the ex-ante heterogeneity. Building on this,
% Insert the Cauchy Schwarz lower bound stuff here

Their empirical strategy is guided by the following conceptual framework. First, they illustrate how the observed duration dependence can decomposed into the true duration dependence and the selection effect. For this purpose, we follow their notation and denote $T_{i,d}$ as the individual job finding probability. $F_{i,d}$ is the observed realization of job finding and $Z_{i,d}$ indicates the perceived job finding rate.

\begin{align}
	\mathbf{E}_d (T_{i,d}) - \mathbf{E}_{d+1} (T_{i,d+1}) = \mathbf{E}_d (T_{i,d} - T_{i,d+1}) + \frac{cov_d (T_{i,d},T_{i,d+1})}{1 - \mathbf{E}_d (T_{i,d})}
\end{align}

The decomposition reveals that the observed duration dependence would coincide with the true duration dependence without dynamic selection $cov_d (T_{i,d},T_{i,d+1}) > 0$ where the actual  job finding rates in adjacent periods would not be correlated with each other. The dynamic selection effect can in itself be decomposed into 

\begin{align}
	cov_d (T_{i,d},T_{i,d+1}) = var_d (T_{i,d}) -  cov_d (T_{i,d},T_{i,d} - T_{i,d+1})
\end{align}

Building on equation 1, this emphasizes that if all of the heterogeneity in job finding probability is fully permanent  and subsequently $cov_d (T_{i,d},T_{i,d} - T_{i,d+1}) = 0$, the role of the dynamic selection is fully captured by the variance of job finding $var_d (T_{i,d})$. Conversely, when all of the heterogeneity of job finding probabilities is fully transitory the terms on the RHS coincide and, subsequently, the selection effect does not contribute to the observed duration dependence. 

Next, the authors exploit the information about the perceived job finding probabilities elicited in the SCE. If the job finding probabilities would be unbiased, then the covariance between realized job finding and  perceived job finding would identify, the variance in ex-ante probabilities  ??? This is surely not properly understood


% Here I need some math
% Preferably I also understand it
\autoref{fig:percep_hist} depicts the histogram of perceptions regarding the probability to find a job in 3 month. The left panel replicates the figure from the paper whereas the right panel uses my own data. Apart from the upper tail of the distribution, the distributions are quite similar. Additionally, both figures can illustrate the bunching where respondents bunch at prominent numbers such as 50\% or 100\%. 

\begin{figure}[!htbp] \centering
	\includegraphics[width=\textwidth]{../../../bld/figures/histogram_perceptions_figure_1.png}
	\caption{Histogram of the Perceptions of U-E 3-Month transition rate}
	\label{fig:percep_hist}
\end{figure}

% Finding 1 
% Optimistic bias
To make the conceptual framework applicable, the elicited job finding probability needs to have predictive power for the realized job finding rate. As seen in \autoref{fig:actual_by_bin} there exists a positive relationship between the actual job finding rate and the elicited probabilities. Additionally, the graph also indicates that respondents in the lower two bins tend to, on average, exhibit pessimistic bias, as their perceived job finding rate is lower than the average job finding rate of their group. Conversely, respondents in the fourth bin and beyond tend to exhibit, on average,  a substantial optimistic bias towards their probability of finding a job. 
% I would like to have another reference here
These findings are consistent with ..., \textcite{Spinne2015} and  \textcite{Baleer2021} which have also documented the existence and sign of such an optimistic bias. Although the different samples tend to behave similarly, differences in the composition of the bins, particularly around the 80\% mark, can lead to some discrepancies.

\begin{figure}[!htbp] \centering
	\includegraphics[width=\textwidth]{../../../bld/figures/jf3mon_per_percbin_figure_2.png}
	\begin{minipage}[center]{\textwidth}
		\caption*{ \scriptsize \textbf{Notes:} All results are based on survey weights. The error bar depicts the respective 95\% confidence interval.}
	\end{minipage}
	\caption{Realized Job Finding Probability by Bin of Perceived Probability}
	\label{fig:actual_by_bin}
\end{figure}

% Finding 2
% Perceptions have significant predictional value for actual job finding
Furthermore, Panel A of \autoref{tab:realized_perc_tabl2} demonstrates the strong predictive capacity of the elicited beliefs, as shown in \autoref{fig:actual_by_bin}. The coefficients based on the own data are presented in the odd-numbered columns, while their counterparts from the authors are located immediately adjacent to them for each specification. All the regressions in this analysis employ weighted least squares (WLS) methodology.
% Go trough line by line (almost)
The results presented in (1) and (2) indicate that in a univariate model, the elicited belief is significant at the 1 percent level. Comparing these results to columns (3) and (4) of the table, we can observe that the predictive power of the belief, as measured by the $R^2$, is nearly as strong as when all demographic observables, such as education, gender, and age, are used to predict the likelihood of the respondent transitioning out of unemployment. Furthermore, the final two columns of Panel A suggest that the predictive power of the elicited belief decreases over the course of the unemployment spell. Overall, the coefficients based on the survey data in Panel A are very similar in magnitude to those reported in the paper.
% Panel B makes only sense after the bootstrap table
\begin{table}[!htbp] \centering 
\tiny
\caption{Linear Regressions of Realized Job Finding Rates on Elicitations} 
\label{tab:realized_perc_tabl2}
\input{../../../bld/tables/tab_2_transition_rate_percep.tex}
\begin{minipage}[center]{0.9\textwidth}
	\caption*{\tiny \textbf{Notes:} All regression use survey weights. The even columns are using the authors data and the uneven columns the own data. 
		Standard errors (in parentheses) are clustered on the individual level. *, **, and *** denote significance at the 10, 5, and 1 percent level.}
\end{minipage}
\end{table}

% Finding 3 
% Identifying the heterogeneity in true job finding rate

Based on the Cauchy-Schwarz inequality with biased beliefs, we can use the previously analyzed relationship between the beliefs and the realized transition rates to compute the lower bound of the true job finding variance. \autoref{tab:lb_variance_table3} depicts the lower bounds computed with the respective specifications. The values of the different specifications are remarkably close we can observe that the lower bound can be tightened by using both elicitations and the observable individual characteristics. 
% Interpet it using the equations

% Also interpret Panel B of Table 2

\begin{center}
	\input{../../../bld/tables/tab3_bootstrap_lower_bound_variance.tex}
\end{center}

% FInding 4
% Optimisitc bias increase over unemployment duration
\begin{figure}[H]
	\begin{subfigure}[b]{0.5\textwidth}
		\centering
		\caption*{\scriptsize{\textbf{Panel A:} Author Data}} \vspace{-.25cm}
		 \includegraphics[width=\textwidth]{../../../bld/figures/perc_vs_real_jf3mon_figure_3_panel_a.png}
		
	\end{subfigure}
	\begin{subfigure}[b]{0.5\textwidth}
		\centering
		\caption*{\scriptsize{\textbf{Panel B:} Own Data}} \vspace{-.25cm}
		 \includegraphics[width=\textwidth]{../../../bld/figures/perc_vs_real_jf3mon_figure_3_panel_b.png}
	\end{subfigure}
\begin{minipage}[center]{\textwidth}
	\caption*{ \scriptsize \textbf{Notes:} All results are based on survey weights. The error bar depicts the respective 95\% confidence interval.}
\end{minipage}
  \caption{Perceived vs. Actual Job Finding Probabilities}
\label{fig:perv_vs_real_udur}
\end{figure}

\autoref{fig:perv_vs_real_udur} shows how the realized and perceived job finding rate evolve throughout the length of the unemployment duration. It has to be noted that by constructing the confidence intervals for the figures there were minor discrepancies between my implementation and the authors implementation which I could not resolve. However, no qualitative or quantitative results is impacted by this difference. Panel A and Panel B both show the initially when the respondents are only unemployed between 0-3 months the beliefs are very accurate. However, throughout the unemployment spell, it appears that the unemployed do not revise their perceptions sufficiently downwards. Whereas the difference between bins 2 and bin 3 in the actual job finding rate is significant, this does not hold true for the beliefs. 
% Here I need more data to foolproof the things I am claiming 

% Finding 5
% Controlling for spell fixed effects,  it vanishes =>
\begin{table}[!htbp] \centering 
\caption{Linear Regressions of Elicitations on Unemployment Duration}
\label{tab:realized_perc_table4}
\input{../../../bld/tables/tab_4_percep_udur.tex}
\begin{minipage}[center]{.9\textwidth}
	\caption*{ \scriptsize \textbf{Notes:} All regression use survey weights. The even columns are using the authors data and the uneven columns the own data. Standard errors (in parentheses) are clustered on the individual level. *, **, and *** denote significance at the 10, 5, and 1 percent level.}
\end{minipage}
\end{table}

\section{Conclusion}\label{sec:conclusion}

\section{References}

\printbibliography


\end{document}
