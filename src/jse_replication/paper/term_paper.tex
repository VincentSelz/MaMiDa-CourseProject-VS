% !TeX spellcheck = en_US
% !TeX document-id = {2870843d-1baa-4f6a-bd0a-a5c796104a32}
% !BIB TS-program = biber
% !TeX encoding = UTF-8

\documentclass[11pt,a4paper,leqno]{article}

\usepackage{a4wide}
\usepackage[T1]{fontenc}
\usepackage[utf8]{inputenc}
\usepackage{float, afterpage, rotating, graphicx}
\usepackage{epstopdf}
\usepackage{longtable, booktabs, tabularx}
\usepackage{fancyvrb, moreverb, relsize}
\usepackage{eurosym, calc}
% \usepackage{chngcntr}
\usepackage[flushleft]{threeparttable}
\usepackage{amsmath, amssymb, amsfonts, amsthm, bm}
\usepackage{caption}
\usepackage{mdwlist}
\usepackage{xfrac}
\usepackage{setspace}
\usepackage{xcolor}
\usepackage{subcaption}
\usepackage{minibox}
\usepackage[hidelinks]{hyperref}
\def\sectionautorefname{Section}
% \usepackage{pdf14} %Enable for Manuscriptcentral
% \usepackage{endfloat} %Enable to move tables / figures to the end. Useful for some submissions.

% BIB SETTINGS
\usepackage[
backend=biber,
giveninits=true,
maxnames=30,
maxcitenames=20,
uniquename=init,
url=false,
style=authoryear,
]{biblatex}
\addbibresource{refs.bib}
\setlength\bibitemsep{0.3cm} % space between entries in the reference list


\widowpenalty=10000
\clubpenalty=10000

\setlength{\parskip}{1ex}
\setlength{\parindent}{0ex}
\setstretch{1.5}

\title{Towards a replication of \textcite{MST2021}:  \\ Using the Survey of Consumer Expectations\thanks{Vincent Selz, Goethe University. Email: \texttt{vincent[dot]selz[at]stud[dot]uni-frankfurt[dot]de}. } }

\author{Vincent Selz \\ 7707241}

\date{
	\today
}

\begin{document}

\clearpage
\maketitle
\thispagestyle{empty}

\clearpage
\section{Introduction}
\setcounter{page}{1}
This term paper aims to replicate elements of \textcite{MST2021}. They leverage the Survey of Consumer's Expectation (henceforth SCE) to investigate job seekers perception of their employment prospects. Specifically, they seek to disentangle the two channels contributing to the observed negative duration dependence: true duration dependence, which involves a decline in the individual job finding rate over the unemployment spell, and the selection effect, which refers to more employable workers exiting unemployment sooner.

To achieve this goal, this term paper will first compare the results obtained through my own calculations with those presented in \textcite{MST2021}. Next, I am using the freely available SCE data to compare the replicated results to this slightly different sample. 
This two-pronged approach has the advantage of both verifying the methods used to arrive at the results and ensuring that the findings are not influenced by data selection.

% Data Strategy
In order to promote transparency and replicability, the entire project has been stored on Github and utilizes the workflow management system, \textit{pytask}, developed by \textcite{Raabe2020}.  Following the instructions provided on Github, it should be possible for anyone  to replicate the results.

\autoref{sec:data} will introduce the SCE and follows the outline given to us for the presentations. Then,
\autoref{sec:paper} will present my work regarding the replication of the reduced form results of the paper and, finally,  \autoref{sec:conclusion} concludes.

\section{Data} \label{sec:data}
% Should be 2/3 pages and can be concluded with Table 1 from MST
% Quick Overview
The SCE is a monthly survey conducted by the Federal Reserve Bank of New York since 2013 that aims to gather information on consumer expectations. \textcite{SCEOverview} provide a comprehensive overview over the SCE. The survey is based on a rotating panel of about 1,300 households, and respondents are asked a range of questions about their expectations for aggregate and individual economic conditions. The data collected from the survey is used to help inform monetary policy decisions and to better understand consumer behavior and economic trends.

% Details on the sample
The sample of the SCE is based on the Consumer Confidence Survey (CCS) which is a mail survey which selects a new random sample each month on the basis of the universe of U.S. Postal Service addresses and targets the household heads. The SCE consists of the subset of CCS respondents who indicated a willingness to participate in the SCE. One condition, however, is the availability of internet and an email address. From the CCS respondents who signaled a willingness to participate, the sample is drawn via a stratified random sampling approach. This stratification spans the following dimensions: income, gender, race/ethnicity and Census Division. Due to the internet based approach, people under the age of 30 are over sampled \parencite{SCEOverview}. Weights are provided to make the sample representative for the population of U.S. household heads. They are based on region, age, education and income and are targeting the Census population estimates from the American Community Survey. 

% SCE and MS
Another prominent consumer survey is the Michigan Survey of Consumers (MSC).
The SCE and the MSC both aim to collect data on consumer expectations, but the SCE focuses on gathering rich, high-frequency data. Compared to the MSC, the SCE has several advantages. Firstly, it covers a broader range of topics and gathers detailed information on areas such as expectations for income growth and home prices. Secondly, the SCE asks questions related to the probability of future events and can also measure uncertainty around the expectation. The broader range of topics covered and detailed information gathered by the SCE make it a more comprehensive tool for analyzing consumer expectations and their potential impact on the economy.

%  How to get the data % Data qualities/ Pecularities
The Federal Reserve Bank of New York's website publishes the results of the SCE on a monthly basis, which are available publicly without requiring a subscription.  All datasets, along with the survey questionnaire, are readily provided. The data is supplied in several excel-files which is convenient but has the disadvantage of consuming comparatively a lot of storage. The variable names are consistent across waves and available in a panel format. The data coding follows the specifications provided in the questionnaire and is easy to comprehend. In addition to the core module, the SCE offers several supplements such as the SCE Credit Access Survey, SCE Public Policy Survey and SCE Household Spending Survey. They are fielded quarterly as a rotating module of the SCE. 

% Density forecasts
Extending the methodology proposed by \textcite{Manski}, the SCE uses three types of questions to elicit probabilistic expectations. The first type of question asks about binary outcomes, such as whether the personal income will increase over the next twelve months. The second type of question asks for point estimates of continuous outcomes, such as the percentage increase or decrease in earnings over the next twelve months. The third type of question asks respondents to give density forecasts, where they have to assign percentage points over a range of bins for the continuous outcomes. Density forecasts provide information beyond just the first moments of the distribution, allowing for the measurement of uncertainty about economic conditions at the individual level. Leveraging this information can provide a new way to test macroeconomic models using micro data and to understand the impact of uncertainty on economic decisions. 

% This feels a little bit bumpy for now
For instance, \textcite{BFKL2018} utilize the SCE to empirically investigate the effect of uncertainty on economic choices. First, they show that there is substantial heterogeneity across households regarding expectations. Second, they document that uncertainty in expectation can predict choices. In particular, they find that more uncertain expectations lead to precautionary behavior. 
%Because in models where certainty equivalence is absent, decisions made today are influenced not only by the expectation of the future wage but also the uncertainty associated with it. Utilizing the subjective density forecasts, in contrast to the rational expectation, can yield a more detailed picture of economic decision-making. This can have implications for differential policies aimed at improving welfare. 

% Check das nochmal anständig und schreib es nochmal anständig.
\textcite{Baleer2021} use the SCE to measure labor risk for various demographic groups. They find substantial optimistic bias which are heterogeneous throughout different groups. Based on this, they show that differences in biases can help explain savings behavior across individuals which is a quantitatively important channel in wealth inequality. 
% One more paper would be great

% Probably easiest to stick quite close to what they are saying about their strategy
\textcite{MST2021} use the elicited beliefs of job seeker's to investigate the negative duration dependence of unemployment. For this purpose, they use the following two questions administered to job seekers in the SCE:
\begin{center}
	\textit{What do you think is the percent chance that within the coming 12 months, you will find a job that you will accept, considering the pay and type of work?}
	\bigskip
	
	\textit{And looking at the more immediate future, what do you think is the percent chance that within the coming 3 months, you will find a job that you will accept, considering the pay and type of work?}
\end{center}

Along with other information, respondents in the survey were asked to report their duration of unemployment. However, self-reports during unemployment spells can be inconsistent.  One major advantage of the SCE is that it experiences low attrition (58\% of all respondents complete all 12 monthly interviews \parencite{SCEOverview}). This allows the authors to construct unemployment durations themselves and thus avoid relying solely on possibly inconsistent self-reports.  Moreover, by tracking individuals over the course of a year, the SCE allows the authors to infer the realized exit rates from unemployment. 
 
     \begin{table}[!htbp] 
 	\centering 
 	\caption{Summary Statistics} 
 	\label{tab:summary_stats} 
 	\input{../../../bld/tables/tab1_summary_statistics_sce.tex}
 	\begin{minipage}[center]{0.7\textwidth}
 		\caption*{\footnotesize \textbf{Notes:} Survey weights are used for all calculations. (3) replicates the reported statistics from the paper exactly. (2) corrects for the error reported in the text and (1) is displaying the summary statistics for the SCE sample.}
 	\end{minipage}
 \end{table}
 
 %Summary statistics
 \autoref{tab:summary_stats} compares the summary statistics of the authors data to the data from the website.\footnote{
 	For simplicity, the data downloaded from the website and cleaned by myself is referred to as my \textit{own} data from now onwards. }  The sample used in the comparison is restricted to unemployed people aged between 20 and 65 who have answered the belief questions. 
 It seems that there is a minor error in the authors calculations. Specifically, when calculating the weighted proportions of the groups, they used the (id-date)-unit instead of the individual level. Hence, they count each individual as many time as they participated in the survey. Upon closer examination of columns (2) and(3), it appears that the sample in the paper is younger and less educated than originally reported, with an even larger share of females. To investigate the extent of the differences between the data used by the authors and my own data, we compare (1) and (2). While the sample used by the authors is larger, the aggregate differences seem to be minor. In fact, the difference between the job finding rate for long-term unemployed and short-term unemployed seems to be even more pronounced in my own data.  

 % Data differences
Based on the number of observations presented in \autoref{tab:summary_stats}, it seems that I was not able to obtain the exact same dataset as the authors. As shown in \autoref{fig:comp_over_time}, the authors had access to earlier data that is, as far as I know, not publicly available. This explains the discrepancies observed during the initial periods. After that the number of responses in subsequent waves is quite similar. Unfortunately, it was not possible to match respondents based on their IDs since the author's dataset uses a different ID structure that cannot be directly linked to the data available on the website.

\begin{figure}[!htbp] \centering
	\includegraphics[width=0.75\textwidth]{../../../bld/figures/data_comparison_over_time.png}
	\begin{minipage}[center]{0.75\textwidth}
		\caption*{\footnotesize \textbf{Notes:} The figure displays the number of respondents in each month for the two data sources.}
	\end{minipage}
  \caption{Data Comparison Over Time}
\label{fig:comp_over_time}
\end{figure}

\section{Paper} \label{sec:paper}

\subsection*{Conceptual Framework}
Understanding the determinants of negative duration dependence of unemployment on the job finding rate remains a topic of long-standing discussion within labor economics.  One possible explanation for this phenomenon is that skills deteriorate over the course of an unemployment spell. Another possibility is that employers screen potential employees based on their duration of unemployment, resulting in lower job finding rates for those who have been unemployed for longer periods. In this approach, the job finding rate declines at the individual level over the unemployment spell. 
Alternatively, some scholars view duration dependence as a sorting mechanism, where more employable individuals find jobs more quickly, leaving the less employable individuals to select into long-term unemployment. However, the primary challenge in understanding duration dependence is that the job finding rate is unobservable, and only the binary outcome whether someone finds a job or not can be observed. 

The authors approach this literature from a new angle. They leverage the elicited beliefs of job seekers and combine them with actual job finding to identify the true heterogeneity of job finding and separate the dynamic selection effect from true (individual) duration dependence. Their empirical strategy is guided by the following conceptual framework. For this purpose, we follow their notation and denote $T_{i,d}$ as the (unobserved) individual job finding probability. $F_{i,d}$ as the observed realization of job finding and $Z_{i,d}$ as the perceived job finding rate. First, they show the decomposition of observed duration dependence into the true duration dependence in job finding and the dynamic selection of job seekers. 
\begin{align}
	\mathbf{E}_d (T_{i,d}) - \mathbf{E}_{d+1} (T_{i,d+1}) = \mathbf{E}_d (T_{i,d} - T_{i,d+1}) + \frac{cov_d (T_{i,d},T_{i,d+1})}{1 - \mathbf{E}_d (T_{i,d})}
\end{align}

The first component on the right-hand side captures the true duration dependence, which reflects that true job finding rate decreases over the time in unemployment. 
% This part does sound a little bit weird as of now
The second component measures the dynamic selection effect, which arises when job seekers who have high (or low) job finding rates in period $d$ are more (or less) likely to find a job in period $d+1$. This effect is captured by the covariance on the right-hand side, which can be decomposed further as follows:
\begin{align}
	cov_d (T_{i,d},T_{i,d+1}) = var_d (T_{i,d}) -  cov_d (T_{i,d},T_{i,d} - T_{i,d+1})
\end{align}

Equation 2 shows that the selection effect depends on the heterogeneity of job finding rates across individuals and time. If all of the heterogeneity in job finding probability is fully permanent, then the role of the dynamic selection effect is fully captured by the variance of job finding rates. In contrast, when all of the heterogeneity is fully transitory, the selection effect does not contribute to the observed duration dependence, since the covariance term is equal to the variance term on the right-hand side.

To separate the role of these forces empirically, the authors use the job seekers' perceived job finding probability from the SCE. They argue that the predictive value of individuals' elicited beliefs can help uncover the heterogeneity in true job finding probabilities, even when beliefs are subject to bias and elicited with error. 
Following \textcite{Morrison2019}, the authors use the Cauchy-Schwarz inequality to bound the variance in job finding probabilities using the covariance between elicited beliefs and ex-post job finding realizations:
\begin{align}
	var_d (T_{i,d}) \geq   \frac{cov_d (Z_{i,d},F_{i,d} )^{2}}{var_d (Z_{i,d}) }
\end{align}
The authors note that, holding the variance in elicitations constant, a larger covariance between ex-ante elicitation and ex-post realization translates towards a larger ex-ante variance in the true job finding rate. The authors further tighten this bound by using multiple elicitations in job seekers' beliefs. This relationship holds true even for specifications that include other observables that are predictive of ex-post job finding, and it can be used to estimate a non-parametric lower bound on the heterogeneity.

\subsection*{Perceptions about Job Finding}

In \autoref{fig:percep_hist}, the histogram illustrates the distribution of job seekers' perceptions of the likelihood of finding employment within three months. The left panel of the figure replicates the original histogram using the authors' data, while the right panel displays the histogram using my own data. At first glance, the two histograms appear quite similar, especially in the bins up to 50\%, which look almost identical. However, there seem to be some discrepancies in the relative frequencies of the bins between 60-80\%. Additionally, both figures can illustrate bunching where respondents bunch their perceptions at prominent numbers such as 50\% or 100\%. 

\begin{figure}[!htbp] \centering
	\includegraphics[width=\textwidth]{../../../bld/figures/histogram_perceptions_figure_1.png}
	\caption{Histogram of the Perceptions of U-E 3-Month transition rate}
	\label{fig:percep_hist}
\end{figure}

% Finding 1 
\subsection*{Predictive Beliefs}
To make the conceptual framework applicable, the elicited job finding probability needs to have predictive power for the realized job finding rate. In order to obtain non-binary measures for the realized job finding, the authors use bins of width 10\% and compute the respective average job finding probability for each bin. \autoref{fig:actual_by_bin} depicts this relationship for each data source and compares it against the rational benchmark where the perception and the realized job finding coincide. The graph shows that there exists a positive relationship between the actual job finding rate and the elicited probabilities. Additionally, it indicates that respondents in the lower two bins tend to, on average, exhibit pessimistic bias, as their perceived job finding rate is lower than the average job finding rate of their group. Conversely, respondents in the fourth bin and beyond tend to exhibit, on average,  substantial optimistic bias towards their probability of finding a job. 
These findings are consistent with \textcite{Spinne2015} and  \textcite{Baleer2021} which have also documented the existence and sign of such an optimistic bias. Moreover, we can compare the different data sources. Although most point estimates are quite similar, there are some discrepancies between the two data sources. In particular, in bins, where have seen above that we have seen composition differences, the optimistic bias are even more pronounced for my own data. Nevertheless, all the qualitative findings stated above hold for both data sources which is reassuring.

\begin{figure}[!htbp] \centering
	\includegraphics[width=\textwidth]{../../../bld/figures/jf3mon_per_percbin_figure_2.png}
	\begin{minipage}[center]{\textwidth}
		\caption*{ \scriptsize \textbf{Notes:} All results are based on survey weights. The error bar depicts the respective 95\% confidence interval.}
	\end{minipage}
	\caption{Realized Job Finding Probability by Bin of Perceived Probability}
	\label{fig:actual_by_bin}
\end{figure}

% Finding 2
% Perceptions have significant predictional value for actual job finding
Furthermore, Panel A of \autoref{tab:realized_perc_tabl2} demonstrates the strong predictive capacity of the elicited beliefs, as shown in \autoref{fig:actual_by_bin}. The coefficients based on the own data are presented in the odd-numbered columns, while their counterparts from the authors are located immediately adjacent to them for each specification. All the regressions in this analysis employ weighted least squares (WLS) methodology.
% Go trough line by line (almost)
The results presented in (1) and (2) indicate that in a univariate model, the elicited belief is significant at the 1 percent level. Comparing these results to columns (3) and (4) of the table, we can observe that the predictive power of the belief, as measured by the $R^2$, is nearly as strong as when all demographic observables, such as education, gender, and age, are used to predict the likelihood of the respondent transitioning out of unemployment. Furthermore, the final two columns of Panel A suggest that the predictive power of the elicited belief decreases over the course of the unemployment spell. Overall, the coefficients based on the survey data in Panel A are very similar in magnitude to those reported in the paper.

\begin{table}[!htbp] \centering 
\tiny
\caption{Linear Regressions of Realized Job Finding Rates on Elicitations} 
\label{tab:realized_perc_tabl2}
\input{../../../bld/tables/tab_2_transition_rate_percep.tex}
\begin{minipage}[center]{0.9\textwidth}
	\caption*{\tiny \textbf{Notes:} All regression use survey weights. The even columns are using the authors data and the uneven columns the own data. 
		Standard errors (in parentheses) are clustered on the individual level. *, **, and *** denote significance at the 10, 5, and 1 percent level.}
\end{minipage}
\end{table}

\subsection*{Lower Bound of ex-ante Heteregeneity in Job Finding}

Based on the Cauchy-Schwarz inequality stated above, the authors can use the previously analyzed relationship between the beliefs and the realized transition rates to compute the lower bound of the true job finding variance. \autoref{tab:lb_variance_table3} depicts the lower bounds computed with the respective specifications. The first row depicts the lower bound from equation 3. In the subsequent rows, the lower bound is tightened by using, first, the 12-month elicitation and, then, demographic characteristics. The results are quite similar, but the lower bound itself lacks interpretability without further context.

For the interpretation, the authors rely on their conceptual framework. When assuming that heterogeneity is fully persistent, $cov_d (T_{i,d},T_{i,d+1}) = var_d (T_{i,d})$ holds. Using the lower bound computed previously, the authors can now plug in $cov_d (T_{i,d},T_{i,d+1})$ into equation 1. This allows them to estimate the upper bound of true duration dependence using only the job finding differences between two adjacent periods. The job finding estimates can be retrieved from \autoref{fig:perv_vs_real_udur}, which documents that the difference between the realized job finding rate withing (0-3) months and within (4-6) months of the unemployment spell is 14.3\%. Using the lower bound based on the 3-month elicitations, it can account for 8.6 percentage points of the decline in job finding rate. This implies that true duration dependence can contribute at most to 39\% of the observed decline. If, we use the lower bound variance based on both elicitations and the controls this would, in fact, leave no role for true duration dependence in this conceptual framework. Both these results are larger than the results the authors obtain because they observe a larger difference in the realized job finding rate in their sample.

\begin{center}
	\input{../../../bld/tables/tab3_bootstrap_lower_bound_variance.tex}
\end{center}

It has to be noted that for the computation of the lower bounds, the authors use the whole sample and when restricting this sample to, for example, short-term unemployed, the estimates for the lower bounds are substantially lower(Appendix Table D4 of their paper). Additionally, the assumption that the heterogeneity of job finding is fully persistent is very restrictive. To explore the persistence of individual job finding rates, the authors assess how predictive the beliefs are when looking at job seekers who remain unemployed for at least 4 month. Panel B of \autoref{tab:realized_perc_tabl2} shows that lagged beliefs have a highly significant impact on the probability of finding a job of similar magnitude to the non-lagged beliefs, suggesting that a significant portion of the variation captured by the elicited beliefs is driven by persistent differences.

% FInding 4
\subsection*{Optimistic Biases over Unemployment Duration}

\begin{figure}[H]
	\begin{subfigure}[b]{0.5\textwidth}
		\centering
		\caption*{\scriptsize{\textbf{Panel A:} Author Data}} \vspace{-.25cm}
		 \includegraphics[width=\textwidth]{../../../bld/figures/perc_vs_real_jf3mon_figure_3_panel_a.png}
		
	\end{subfigure}
	\begin{subfigure}[b]{0.5\textwidth}
		\centering
		\caption*{\scriptsize{\textbf{Panel B:} Own Data}} \vspace{-.25cm}
		 \includegraphics[width=\textwidth]{../../../bld/figures/perc_vs_real_jf3mon_figure_3_panel_b.png}
	\end{subfigure}
\begin{minipage}[center]{\textwidth}
	\caption*{ \scriptsize \textbf{Notes:} All results are based on survey weights. The error bar depicts the respective 95\% confidence interval.}
\end{minipage}
  \caption{Perceived vs. Actual Job Finding Probabilities}
\label{fig:perv_vs_real_udur}
\end{figure}

\autoref{fig:perv_vs_real_udur} shows how the realized and perceived job finding rate evolve throughout the length of the unemployment duration. There were minor discrepancies between my implementation and the authors implementation during the construction of the confidence intervals. My calculation, however, are based on the same code with which I was able to fully replicate the results found in \autoref{fig:actual_by_bin}. The panels suggest that, initially, respondents' beliefs are very accurate when they are unemployed for up to three months. After three months the negative duration dependence in actual job finding can be observed, but the unemployed do not revise their perceptions sufficiently downwards, leading to an optimistic bias. 

\autoref{tab:realized_perc_table4} reports regression results using different specifications. In the first two columns, only the first elicitation for each unemployment spell is used. It shows that increasing the unemployment duration by one month, decreases the elicited job finding probability by 0.6 percentage points. 
When all elicitations and observable characteristics are used (columns 2-6), the magnitude is smaller but remains highly significant on the 1\% level. However, when controlled for by spell fixed effects in columns 7 and 8, the coefficient becomes slightly positive, although not significant, indicating that the impact is not driven by true duration dependence.

% Finding 5
% Controlling for spell fixed effects,  it vanishes =>
\begin{table}[!htbp] \centering 
\caption{Linear Regressions of Elicitations on Unemployment Duration}
\label{tab:realized_perc_table4}
\input{../../../bld/tables/tab_4_percep_udur.tex}
\begin{minipage}[center]{.9\textwidth}
	\caption*{ \scriptsize \textbf{Notes:} All regression use survey weights. The even columns are using the authors data and the uneven columns the own data. Standard errors (in parentheses) are clustered on the individual level. *, **, and *** denote significance at the 10, 5, and 1 percent level. In column (7), the standard errors are larger than the ones reported in the table. }
\end{minipage}
\end{table}

\section{Conclusion}\label{sec:conclusion}

Based on my analysis, I find that the results presented in the paper are robust and largely consistent with the authors' narrative. I was able to replicate most of their reduced form findings and confirm that their data is of high quality.

The paper makes an important contribution to the literature on duration dependence by showing that perceived job finding probabilities are overly optimistic and that true duration dependence is not a significant driver of the observed duration dependence. The authors' approach of using survey data to estimate perceptions and compare them to actual job finding rates is innovative and provides a unique perspective on the issue.

Overall, the paper's findings have important implications for policymakers and individuals seeking employment. The results suggest that job seekers may benefit from more accurate information about their job prospects, particularly as their unemployment spell lengthens. Policymakers could use this information to design more effective programs to support the unemployed, and to help them transition back into the labor force more quickly.

\newpage

\printbibliography


\end{document}
